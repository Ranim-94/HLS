




% Macros File

% Macros Example
\def\labelaxes{Remember to include some suitable labeling for the axes and the units used in measurements.}

% 2nd way for defnining macros: the new command
% 1st parameter: name of the new command
% 2nd paramter: how many inputs this command needs, in this case only 1
% 3rd parameter: what this new command named \tbi do

% command 1
\newcommand{\tbi}[1]{\textbf{\textit{#1}}}

% command 2
\newcommand{\uti}[1]{\underline{\textit{#1}}}

% Command for Picture without a label

\newcommand{\pic}[3]{\begin{figure}[h]
\centering
\includegraphics[width = 0.7\textwidth, frame]{#1}
\caption{#2}
\end{figure}}

% wirte def above = in equations
\newcommand\myeq{\stackrel{\mathclap{\normalfont\mbox{def}}}{=}}

% big dot 
\makeatletter
\newcommand*\bigcdot{\mathpalette\bigcdot@{.5}}
\newcommand*\bigcdot@[2]{\mathbin{\vcenter{\hbox{\scalebox{#2}{$\m@th#1\bullet$}}}}}
\makeatother


% Math Opeartor
\DeclareMathOperator*{\argmax}{argmax}
\DeclareMathOperator*{\argmin}{argmin}
